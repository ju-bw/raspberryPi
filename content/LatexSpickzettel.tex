% ju -- https://bw1.eu -- 25-April-18  -- kap1.tex 
%\chapter{Kapitel}
%
\section{\LaTeX - Spickzettel}\label{sec:LaTeX-Spickzettel}

\subsection{Blindtext}\label{sec:blindtext}

\blindtext[1] 

\blindtext[1]

\subsection{Flattersatz versus Blocksatz}\label{sec:Flattersatz-versus-Blocksatz}

\begin{flushleft}
\blindtext[1]
\end{flushleft}

\begin{flushright}
\blindtext[1]
\end{flushright}

\begin{center}
\blindtext[1]
\end{center}

\newpage

\subsection{Gliederung}

% Quellcode Referenz
(\autoref{code:gliederung} Gliederung).
% Quellcode
\lstset{language=[LaTeX]TeX} % C, [LaTeX]TeX, Bash, Python
\begin{lstlisting}[columns=fullflexible,%numbers=left, frame=l, framerule=0.1pt,%
% ======================
  caption={Gliederung},    % Caption anpassen!
  label={code:gliederung}  % Label anpassen!
]% ======================

	% artikel:         Textabstand
	\subsection \subsubsection \paragraph \subparagraph

	% report:
	\chapter \subsection \subsubsection \paragraph \subparagraph

	% * keine Nummerierung
	\subsection*{eins}

	% Sprungmarke
	\label{sec:eins}
\end{lstlisting}

\newpage

\subsection{Quellcode}

Quellcode \verb|Code|.

% Quellcode Referenz
(\autoref{code:dummyCode} dummyCode).  % Anpassen!
% Quellcode
\lstset{language=[LaTeX]TeX} % C, [LaTeX]TeX, Bash, Python
\begin{lstlisting}[columns=fullflexible,%numbers=left, frame=l, framerule=0.1pt,%
% ======================
  caption={dummyCode},    % Caption anpassen!
  label={code:dummyCode}  % Label anpassen!
]% ======================

	/* Quellcode */
\end{lstlisting}

% Quellcode Ausgabe Referenz
(\autoref{code:dummyCode-out} dummyCode Ausgabe).  % Anpassen!
% Quellcode
\lstset{language=[LaTeX]TeX} % C, [LaTeX]TeX, Bash, Python
\begin{lstlisting}[columns=fullflexible,%numbers=left, frame=l, framerule=0.1pt,%
% ======================
  caption={dummyCode Ausgabe},    % Caption anpassen!
  label={code:dummyCode-out}  % Label anpassen!
]% ======================

	% Quellcode Referenz
	(\autoref{code:dummyCode} dummyCode).  % Anpassen!
	% Quellcode
	\lstset{language=[LaTeX]TeX} % C, [LaTeX]TeX, Bash, Python
	\begin{lstlisting}[columns=fullflexible,%numbers=left, frame=l, framerule=0.1pt,%
	% ======================
		caption={dummyCode},    % Caption anpassen!
		label={code:dummyCode}  % Label anpassen!
	]% ======================

		/* Quellcode */
	%end{lstlisting}
\end{lstlisting}

% Quellcode Referenz
(\autoref{code:hallo-ext} hallo.c).    % codename = 
% Quellcode aus ext. Datei
	\lstset{language=C}% C++, [LaTeX]TeX, Bash, Python
	\lstinputlisting[%numbers=left, frame=l, framerule=0.1pt,
	% =====================
	caption={Quellcode in C, hallo.c},    % Caption  anpassen!
	label={code:hallo-ext}  % Referenz anpassen!
	% =====================
	]{content/hallo.c}% ext. Datei

\newpage

\subsection{Querverweise-Referenzen}\label{sec:quer-ref}

(\autoref{sec:schriftstile} Schriftstile).

(\autoref{sec:listen} Listen).

(\autoref{pic:tux1} Tux1).

\begin{enumerate}
	\item zuerst
	\item \label{item:folge} folgend
	\item abschließend
\end{enumerate}

Wir verweisen auf ein Listenelement \autoref{item:folge}.

% Quellcode Ausgabe Referenz
(\autoref{code:quer-ref-out} Querverweise-Referenzen).% Anpassen!
% Quellcode
\lstset{language=[LaTeX]TeX} % C, [LaTeX]TeX, Bash, Python
\begin{lstlisting}[columns=fullflexible,%numbers=left, frame=l, framerule=0.1pt,%
% ======================
	caption={Quellcode in LaTeX, Querverweise-Referenzen},% Anpassen!
	label={code:quer-ref-out}%
	%============================
]
	(\autoref{sec:schriftstile} Schriftstile).
	(\autoref{sec:listen} Listen).
	(\autoref{pic:tux1} Tux1).
	\begin{enumerate}
		\item zuerst
		\item \label{item:folge} folgend
		\item abschließend
	\end{enumerate}
	Wir verweisen auf ein Listenelement \autoref{item:folge}.
\end{lstlisting}

% Tabelle Referenz
(\autoref{tab:label-verweis} Label Querverweis).
% Tabelle
\begin{table}[!hb] % hier einfügen
	%============================
	\caption{Label Querverweis }	% Caption anpassen!
  \label{tab:label-verweis}	% Referenz anpassen!
	%============================
	\centering
	\begin{tabular} {ll}
	\toprule % Inhalt
		Abk. & Beschreibung \\
  \midrule
		sec  & für alle Gliederungsebenen \\
		cha  & oder chap für Kapitel (es kann aber auch sec verwendet werden) \\
		part & für Teile eines Buches (ebenso sec möglich) \\
		fig  & für Abbildungen \\
		tab  & für Tabellen \\
		item & für Aufzählungspunkte \\
		eqn  & für Gleichungen  \\
		fn   & für Fußnoten \\
		code & Listing \\
		pic  & Grafik \\
	\bottomrule
	\end{tabular}
\end{table}

\newpage

\subsection{Zitieren}\label{sec:zitieren}

\begin{quote}
Ein schönes Zitat von einem schlauen Menschen steht den meisten Dokumenten
gut zu Gesicht.
\end{quote}

Fussnote\footnote{Fussnote}.

Google\footnote{\url{https://www.google.de/}}

Anfang ">Anführungszeichen."< Ende

Anfang "`Anführungszeichen."' Ende

Anfang \flqq Anführungszeichen.\frqq Ende

\LaTeX buch \cite{Schlosser201609}.

% Quellcode Referenz
(\autoref{code:zitieren-out} Zitieren).
% Quellcode
\lstset{language=[LaTeX]TeX} % C, [LaTeX]TeX, Bash, Python
\begin{lstlisting}[columns=fullflexible,%numbers=left, frame=l, framerule=0.1pt,%
% ======================
	caption={Quellcode in LaTeX, Zitieren},% Anpassen!
	label={code:zitieren-out}%
	%============================
]
	\begin{quote}
		Ein schönes Zitat von einem schlauen Menschen steht 
		den meisten Dokumenten gut zu Gesicht.
	\end{quote}
	Fussnote\footnote{Fussnote}.
	Anfang ">Anführungszeichen."< Ende
	Anfang "`Anführungszeichen."' Ende
	Anfang \flqq Anführungszeichen.\frqq Ende
	\LaTeX buch \cite{Schlosser201609}.
\end{lstlisting}

\newpage

\subsection{Links}\label{sec:links}\index{Links}

Darstellung einer klickbaren URL: \url{https://www.google.de/}

Text, der auf eine Webseite linkt: \href{https://bw1.eu/}{Meine Webseite}

Emailadresse verlinken: \href{mailto:info@bw1.eu}{Meine E-Mail-Adresse}

auf lokale Datei verlinken: \href{run:/word.docx}{lokale Datei} 

PDF einbinden: \includepdf[pagecommand={\thispagestyle{headings}},
	noautoscale=true,width=0.9\textwidth,offset=0cm -1cm]{content/titelbild.pdf}

PDF einbinden: \\ 

\includegraphics[width=0.9\textwidth]{content/titelbild.pdf}


(\autoref{code:links-out} Links).
% Quellcode
\lstset{language=[LaTeX]TeX} % C, [LaTeX]TeX, Bash, Python
\begin{lstlisting}[columns=fullflexible,%numbers=left, frame=l, framerule=0.1pt,%
% ======================
	caption={Quellcode in LaTeX, Links},% Anpassen!
	label={code:links-out}%
	%============================
]
	Darstellung einer klickbaren URL: \url{https://www.google.de/}

	Text, der auf eine Webseite linkt: \href{https://bw1.eu/}{Meine Webseite}

	Emailadresse verlinken: \href{mailto:info@bw1.eu}{Meine E-Mail-Adresse}

	auf lokale Datei verlinken: \href{run:/word.docx}{lokale Datei} 

	PDF einbinden: \includepdf[pagecommand={\thispagestyle{headings}},
		noautoscale=true,width=0.9\textwidth,offset=0cm -1cm]{content/titelbild.pdf}

	PDF einbinden: \\
	
	\includegraphics[width=0.9\textwidth]{content/titelbild.pdf}
\end{lstlisting}

\newpage

\subsection{Farbe}

Text \textcolor{red}{rot} Text

Text \colorbox{hellesbrombeer}{hellesbrombeer} Text

\colorbox{red!10!white}{10 \% rot, Rest weiß}

\textcolor{meinorange}{farbiger Text}
\textcolor{meinblue}{farbiger Text}
\textcolor{meinred}{farbiger Text}
\textcolor{magenta}{farbiger Text}

\wichtig[meinblue]{wichtiger farbiger Text}
\wichtig[meinred]{wichtiger farbiger Text}
\wichtig[meinorange]{wichtiger farbiger Text}
\wichtig[magenta]{wichtiger farbiger Text}

% Farbbox
\colorbox{meingrey}{Text}
\colorbox{meinorange}{Text}

% bunter Rahmen um eine Formel
\fcolorbox{meinblue}{meingrey}{$a^{2} + b^{2} = c^{2}$}

% Quellcode Referenz
(\autoref{code:farbe-out} Farbe).
% Quellcode
\lstset{language=[LaTeX]TeX} % C, [LaTeX]TeX, Bash, Python
\begin{lstlisting}[columns=fullflexible,%numbers=left, frame=l, framerule=0.1pt,%
% ======================
	caption={Quellcode in LaTeX, Farbe},% Anpassen!
	label={code:farbe-out}%
	%============================
]
	\textcolor{meingreen}{farbiger Text}
	\textcolor{meinblue}{farbiger Text}
	\textcolor{meinred}{farbiger Text}

	\wichtig[meinblue]{wichtiger farbiger Text}
	\wichtig[meinred]{wichtiger farbiger Text}
	\wichtig[meingreen]{wichtiger farbiger Text}

	% Farbbox
	\colorbox{meingrey}{Text}
	\colorbox{meinorange}{Text}

	% bunter Rahmen um eine Formel
	\fcolorbox{meinblue}{meingrey}{$a^{2} + b^{2} = c^{2}$}
\end{lstlisting}

\begin{hinweis}
 Es gibt zwei Möglichkeiten.
 Du schreibst eine for-Schleife, die alle Namen durchgeht und
 abbricht (mit break), wenn ein Name mit D gefunden wurde.
 Du schreibst eine while-Schleife, die der Reihe nach die Namen anschaut,
 solange sie nicht mit D anfangen. Dazu verwendest Du einen Zähler,
 der bei 0 anfängt, und der Reihe nach alle Namen abfragt.
\end{hinweis}

\myInfoBox{
 Es gibt zwei Möglichkeiten.
 Du schreibst eine for-Schleife, die alle Namen durchgeht und
 abbricht (mit break), wenn ein Name mit D gefunden wurde.
 Du schreibst eine while-Schleife, die der Reihe nach die Namen anschaut,
 solange sie nicht mit D anfangen. Dazu verwendest Du einen Zähler,
 der bei 0 anfängt, und der Reihe nach alle Namen abfragt.
}

% Quellcode Referenz
(\autoref{code:hinweis-infobox-out} Hinweis, Infobox).
% Quellcode
\lstset{language=[LaTeX]TeX} % C, [LaTeX]TeX, Bash, Python
\begin{lstlisting}[columns=fullflexible,%numbers=left, frame=l, framerule=0.1pt,%
% ======================
	caption={Quellcode in LaTeX, Hinweis, Infobox},% Anpassen!
	label={code:hinweis-infobox-out}%
	%============================
]
	\begin{hinweis}
	 Es gibt zwei Möglichkeiten.
	 Du schreibst eine for-Schleife, die alle Namen durchgeht und
	 abbricht (mit break), wenn ein Name mit D gefunden wurde.
	 Du schreibst eine while-Schleife, die der Reihe nach die Namen anschaut,
	 solange sie nicht mit D anfangen. Dazu verwendest Du einen Zähler,
	 der bei 0 anfängt, und der Reihe nach alle Namen abfragt.
	\end{hinweis}

	\myInfoBox{
	 Es gibt zwei Möglichkeiten.
	 Du schreibst eine for-Schleife, die alle Namen durchgeht und
	 abbricht (mit break), wenn ein Name mit D gefunden wurde.
	 Du schreibst eine while-Schleife, die der Reihe nach die Namen anschaut,
	 solange sie nicht mit D anfangen. Dazu verwendest Du einen Zähler,
	 der bei 0 anfängt, und der Reihe nach alle Namen abfragt.
	}
\end{lstlisting}

\newpage

\subsection{Tabellen}\label{sec:tabellen}

\begin{tabular}{ccc}
	\toprule
		Leistung & 45 & kWh \\
	\midrule
		Hubraum & $1234$ & $cm^3$ \\
		Preis & 23499 & Euro \\
	\bottomrule
\end{tabular}\\

Text

% Tabellen Referenz
(\autoref{tab:dummyTabelle} dummyTabelle).
% Tabelle
\begin{table}[!hb] % hier
	\centering
	%\setlength{\tabcolsep}{5mm}      % Spaltenlänge fest
	%\begin{tabularx}{\textwidth}{XX} % auto. Spaltenumbruch
	\rowcolors{1}{ }{lightgray!20} % Farbe
	\begin{tabular} {ll}
		\toprule 
		%\rowcolor{orange!90} % Farbe
		%============================
	  \textbf{A} & \textbf{B} \\
	  \midrule
    a1 & a2 \\
    b1 & b2 \\
    c1 & c2 \\
		%============================
		\bottomrule
	%\end{tabularx}
	\end{tabular}
	%============================
	\caption{dummyTabelle}   % Caption anpassen!
	\label{tab:dummyTabelle} % Referenz anpassen!
	%============================
\end{table}

% Quellcode Referenz
(\autoref{code:dummyTabelle-out} dummyTabelle).
% Quellcode
\lstset{language=[LaTeX]TeX} % C, [LaTeX]TeX, Bash, Python
\begin{lstlisting}[columns=fullflexible,%numbers=left, frame=l, framerule=0.1pt,%
% ======================
	caption={Quellcode in LaTeX, dummyTabelle},% Anpassen!
	label={code:dummyTabelle-out}%
	%============================
]
	% Tabellen Referenz
	(\autoref{tab:dummyTabelle} dummyTabelle).
	% Tabelle
	\begin{table}[!hb] % hier
		\centering
		%\setlength{\tabcolsep}{5mm}% Spaltenlänge fest
		\rowcolors{1}{ }{lightgray!20} % Farbe
		%\begin{tabularx}{\textwidth}{XX}% auto. Spaltenumbruch
		\begin{tabular} {ll}
			\toprule 
			%\rowcolor{meinblue!90}% Farbe
			%============================
				\textbf{A} & \textbf{B} \\
			\midrule
				a1 & a2 \\
				b1 & b2 \\
				c1 & c2 \\
			%============================
			\bottomrule
		%\end{tabularx}
		\end{tabular}
		%============================
		\caption{dummyTabelle}% Caption anpassen!
		\label{tab:dummyTabelle}					  % Referenz anpassen!
		%============================
	\end{table}
\end{lstlisting}

% Tabellen Referenz
(\autoref{tab:spalte-fest} Spaltenlänge fest).
% Tabelle
\begin{table}[!hb] % hier
	\centering
	\setlength{\tabcolsep}{5mm}% Spaltenlänge fest
	%\begin{tabularx}{\textwidth}{XX}% auto. Spaltenumbruch
	\rowcolors{1}{ }{lightgray!20} % Farbe 
	\begin{tabular} {ll}
		\toprule 
		%\rowcolor{meinblue!90}% Farbe
		%============================
	    \textbf{A} & \textbf{B} \\
	  \midrule
			a1 & a2 \\
			b1 & b2 \\
			c1 & c2 \\
		%============================
		\bottomrule
	%\end{tabularx}
	\end{tabular}
	%============================
	\caption{Spaltenlänge fest}% Caption anpassen!
	\label{tab:spalte-fest}					  % Referenz anpassen!
	%============================
\end{table}

(Tabelle Longtable).
% Longtable
\rowcolors{1}{ }{lightgray!20} % Farbe
\begin{longtable}{ll}% Spaltenanzahl, l,r,c,p,X
	\toprule
	%\rowcolor{meinblue!90}% Farbe
	\textbf{A} & \textbf{B} \\
	\midrule
	\endfirsthead
	\toprule
	% Tab.-Fortsetzung
	%\rowcolor{meinblue!90}% Farbe
	\textbf{A} & \textbf{B} \\
	\midrule
	\endhead
	% Inhalt
	a1 & a2 \\
	b1 & b2 \\
	c1 & c2 \\
	a1 & a2 \\
	b1 & b2 \\
	c1 & c2 \\
	a1 & a2 \\
	b1 & b2 \\
	c1 & c2 \\
	a1 & a2 \\
	b1 & b2 \\
	c1 & c2 \\
	c1 & c2 \\
	a1 & a2 \\
	b1 & b2 \\
	c1 & c2 \\
	\bottomrule
\end{longtable}

\newpage

\subsection{Abbildungen}\label{sec:abbildungen}

% Bild Referenz
(\autoref{pic:tux1} Linux Pinguin Tux).
% Bild
\begin{figure}[!hb] % hier
	\centering
  \includegraphics[width=0.3\textwidth]{img/logo.pdf}
	% ====================
	\caption[Tux]{">Ein wohlgenährter, glücklicher, rundlicher Pinguin, ist das
									offizielle Maskottchen des freien Betriebssystemkerns Linux."<
	\newline {(Quelle:~Wikipedia)}}% Caption  anpassen!
	\label{pic:tux1}% Referenz anpassen!
	% ===================
\end{figure}

\includegraphics[height=2cm,width=3cm]{img/logo.pdf}
\includegraphics[angle=45, scale=.2]{img/logo.pdf}

% Bild Referenz
(\autoref{pic:dummyAbb} dummyAbb).    % Anpassen!
% Bild
\begin{figure}[!hb]% hier
	\centering
  \includegraphics[width=0.3\textwidth]{img/logo.pdf}
	% ====================
	\caption{dummyAbb}% Caption  anpassen!
	\label{pic:dummyAbb}% Referenz anpassen!
	% ===================
\end{figure}

% Bild Referenz
(\autoref{pic:bildDrehen} Drehen um 45 Grad).    % Anpassen!
% Bild
\begin{figure}[!hb]% hier
	\centering
  \includegraphics[angle=45,width=0.3\textwidth]{img/logo.pdf}
	% ====================
	\caption{Drehen um 45 Grad}% Caption  anpassen!
	\label{pic:bildDrehen}% Referenz anpassen!
	% ===================
\end{figure}

\newpage


% Quellcode Referenz
(\autoref{code:dummyAbb-out} dummyAbb).
% Quellcode
\lstset{language=[LaTeX]TeX} % C, [LaTeX]TeX, Bash, Python
\begin{lstlisting}[columns=fullflexible,%numbers=left, frame=l, framerule=0.1pt,%
% ======================
	caption={Quellcode in LaTeX, dummyAbb},% Anpassen!
	label={code:dummyAbb-out}%
	%============================
]
	% Bild Referenz
	(\autoref{pic:dummyAbb} dummyAbb).    % Anpassen!
	% Bild
	\begin{figure}[!hb]% hier
		\centering
		\includegraphics[width=0.3\textwidth]{img/logo.pdf}
		% ====================
		\caption{dummyAbb}% Caption  anpassen!
		\label{pic:dummyAbb}% Referenz anpassen!
		% ===================
	\end{figure}
\end{lstlisting}

\newpage

\subsection{Scalieren}\index{Skalieren}

Inhalt n-fach scalieren

\vspace{10mm}

\scalebox{3}{Text}
\scalebox{4}{Text}
\scalebox{5}{Text}

\subsection{Rotieren}

Inhalt rotieren - Wert in Grad

\vspace{10mm}

\rotatebox{45}{Text}
\rotatebox{90}{Text}
\rotatebox{180}{Text}


% Quellcode Referenz
(\autoref{code:scal-rot-out} Scalieren und Rotieren).
% Quellcode
\lstset{language=[LaTeX]TeX} % C, [LaTeX]TeX, Bash, Python
\begin{lstlisting}[columns=fullflexible,%numbers=left, frame=l, framerule=0.1pt,%
% ======================
	caption={Quellcode in LaTeX, Scalieren und Rotieren},% Anpassen!
	label={code:scal-rot-out}%
%============================
]
	% Inhalt n-fach scalieren
	\scalebox{3}{Text}
	\scalebox{4}{Text}
	\scalebox{5}{Text}

	% Inhalt rotieren - Wert in Grad
	\rotatebox{45}{Text}
	\rotatebox{90}{Text}
	\rotatebox{180}{Text}
\end{lstlisting}

\newpage

\subsection{Gliederung in Kapitel und Abschnitte}\label{sec:Gliederung-Kapitel-Abschnitte}

(\autoref{tab:Gliederung-Kapitel-Abschnitte} Gliederung in Kapitel und Abschnitte).
% Tabelle
\begin{table}[!hb]% hier 
	\centering
	%\setlength{\tabcolsep}{5mm}% Spaltenlänge fest
  \rowcolors{1}{ }{lightgray!20} % Farbe
	%\begin{tabularx}{\textwidth}{XX}% auto. Spaltenumbruch
	\begin{tabular} {ll}
		\toprule % Inhalt
		%\rowcolor{meinblue!90}% Farbe
		%============================
		\verb|#| & \textbf{Beschreibung} \\
	  \midrule
		\verb|\chapter{}      | & Ein Kapitel\\
		\verb|\section{}      | & Ein Abschnitt\\
		\verb|\subsection{}   | & Ein Unterabschnitt\\
		\verb|\subsubsection{}| & Ein Unter-Unterabschnitt\\
		\verb|\paragraph{}    | & Ein Absatz\\
		\verb|\subparagraph{} | & Ein Unterabsatz\\
		\verb|\subsection*{}  | & Ein unnummerierter Abschnitt\\
		\verb|\subsection[Kurzer Titel]{}| & langer Abschnittstitel\\
		%============================
		\bottomrule
	%\end{tabularx}
	\end{tabular}
	\caption{Gliederung in Kapitel und Abschnitte}	% Caption anpassen!
	\label{tab:Gliederung-Kapitel-Abschnitte}	% Referenz anpassen!
	%============================
\end{table}

\subsection{Schriftstile}\label{sec:schriftstile}

\emph{kursiv}
\textrm{Antiqua}, \textsf{Grotesk}, \texttt{Maschinenschrift},
\textmd{normal}, \textbf{breiter}, \textup{aufrecht}, \textsl{geneigt},
\textit{kursiv}, \textsc{Kapitaelchen}

\subsection{Schriftgrößen}

\tiny{winzig}, \scriptsize{sehr klein}, \footnotesize{klein},
\small{klein}, , \large{gross}, \Large{groesser},
\LARGE{ganz gross}, \huge{riesig}, \Huge{gigantisch} \normalsize{normal}

\subsection{Wortabstände}\label{sec:Wortabstaende}

(\autoref{tab:wortabstand} Wortabstände).
% Tabelle
\begin{table}[!hb] % hier 
	\centering
	%\setlength{\tabcolsep}{5mm}% Spaltenlänge fest
	\rowcolors{1}{ }{lightgray!20} % Farbe
	%\begin{tabularx}{\textwidth}{XX}% auto. Spaltenumbruch
	\begin{tabular} {ll}
	\toprule % Inhalt
	%\rowcolor{meinblue!90}% Farbe
	%============================
	\verb|#| & \textbf{Beschreibung} \\
	\midrule
			\verb|\| & erzeugt Leerstelle \\
			\verb|\@| & kennzeichnet einen Punkt als Satzende \\
			\verb|~| & erzeugt nicht umbrechbare Leerstelle \\
			\verb|\,| & erzeugt nicht umbrechbare Leerstelle \\
			\verb|\quad| & erzeugt einfach vergrößerten Abstand \\
			\verb|\qquad| & erzeugt vierfach vergrößerten Abstand \\
			\verb|\hspace{1cm}| & erzeugt Abstand von 1cm Breite \\
			\verb|\hfill| & fügt so viel Leerraum ein wie möglich \\
			\verb|\smallskip| & vertikaler Abstand\\
			\verb|\medskip| & \\
			\verb|\bigskip| & \\
			\verb|\vspace{1cm}| & \\
			\verb|\vfill| & \\
	%============================
	\bottomrule
	%\end{tabularx}
	\end{tabular}
	\caption{Wortabstände }	% Caption anpassen!
	\label{tab:wortabstand}	% Referenz anpassen!
	%============================
\end{table}

\subsection{Logische Textauszeichnung}\label{sec:LogischeTextauszeichnung}

(\autoref{tab:Textauszeichnung} Logische Textauszeichnung).
% Tabelle
\begin{table}[!hb] % hier 
	\centering
	%\setlength{\tabcolsep}{5mm}% Spaltenlänge fest
	\rowcolors{1}{ }{lightgray!20} % Farbe
	%\begin{tabularx}{\textwidth}{XX}% auto. Spaltenumbruch
	\begin{tabular} {ll}
	\toprule % Inhalt
	%\rowcolor{meinblue!90}% Farbe
	%============================
	\verb|#| & \textbf{Beschreibung} \\
	\midrule
			\verb|\emph{Hervorhebung}| & \emph{Hervorhebung} \\
			 \verb|\url{http://www.dante.de/}| & \url{http://www.dante.de/} \\
			 \verb|\href{https://bw1.eu/}{Meine Webseite} |& \href{https://bw1.eu/}{Meine Webseite} \\
			 \verb|\href{mailto:info@bw1.eu}{info@bw1.eu} |& \href{mailto:info@bw1.eu}{info@bw1.eu} \\
			 \verb|\path{/home/foo/meindok.tex}| & \path{/home/foo/meindok.tex} \\
			 \verb|\path{C:\TEMP\meindok.tex}| & \path{C:\TEMP\meindok.tex} \\
			 \verb|\wort{Text} | & \wort{Text} \\
			 \verb|\fremdwort{Text} |& \fremdwort{Text} \\
			 \verb|Sonderzeichen: \& \% \$ \# \_ \{ \} |& \& \% \$ \# \_ \{ \} \\
			 \verb|\LaTeX|& \LaTeX \\
			 \verb|\dots|& \dots \\
	%============================
	\bottomrule
	%\end{tabularx}
	\end{tabular}
	\caption{Logische Textauszeichnung}	% Caption anpassen!
	\label{tab:Textauszeichnung}	% Referenz anpassen!
	%============================
\end{table}

\subsection{Punkte}\label{sec:Punkte}

(\autoref{tab:punkt} Punkte).
% Tabelle
\begin{table}[!hb] % hier einfügen
	\centering
	%\setlength{\tabcolsep}{5mm}% Spaltenlänge fest
	\rowcolors{1}{ }{lightgray!20} % Farbe
	%\begin{tabularx}{\textwidth}{XX}% auto. Spaltenumbruch
	\begin{tabular} {ll}
	\toprule % Inhalt
	%\rowcolor{meinblue!90}% Farbe
	%============================
	\verb|#| & \textbf{Beschreibung} \\
	\midrule
	\verb|Deutsch: Eins, zwei, ...| & Deutsch: Eins, zwei, ... \\
	\verb|Amerikanisch: One, two,~\dots| & Amerikanisch: One, two,~\dots \\
	%============================
	\bottomrule
	%\end{tabularx}
	\end{tabular}
	\caption{Punkte }	% Caption anpassen!
	\label{tab:punkt}	% Referenz anpassen!
	%===========================
\end{table}

\subsection{Binde- und Gedankenstriche}\label{sec:Binde-Gedankenstriche}

(\autoref{tab:Bindestriche } Binde- und Gedankenstriche).
% Tabelle
\begin{table}[!hb]% hier 
	\centering
	%\setlength{\tabcolsep}{5mm}% Spaltenlänge fest
	\rowcolors{1}{ }{lightgray!20} % Farbe
	%\begin{tabularx}{\textwidth}{XX}% auto. Spaltenumbruch
	\begin{tabular} {ll}
	\toprule % Inhalt
	%\rowcolor{meinblue!90}% Farbe
	%============================	
	\verb|#| & \textbf{Beschreibung} \\
	\midrule
	\verb|n-zu-m-Abbildung| & n-zu-m-Abbildung \\
	\verb|11--19 Uhr| & 11--19 Uhr \\
	\verb|Berlin--Hamburg| & Berlin--Hamburg \\
	\verb|wahr -- oder falsch?| & wahr -- oder falsch? \\
	\verb|true---or false?| & true---or false? \\
	\verb|1, 0, $-$| & 1, 0, $-$ \\
	%============================
	\bottomrule
	%\end{tabularx}
	\end{tabular}
	\caption{Binde- und Gedankenstriche }	% Caption anpassen!
	\label{tab:Bindestriche }	% Referenz anpassen!
	%===========================
\end{table}

\newpage

\subsection{Listen}\label{sec:listen}

%% Liste
\begin{itemize}% Liste Punkt
	\item Text
	\item Text
\end{itemize}

\begin{enumerate}% Liste Aufzählung
	\item Text
	\item Text
\end{enumerate}

\begin{enumerate}% Liste 1.
	\item Text
	\begin{enumerate}% Liste a)
		\item Text
		\item Text
	\end{enumerate}% Liste 2.
	\item Text
	\begin{enumerate}% Liste b)
		\item Text
		\item Text
	\end{enumerate}
\end{enumerate}

\begin{enumerate}[label=(\roman*)]% (i), (ii), (iii) ...
	\item Text
	\item Text
\end{enumerate}

\begin{enumerate}[label={\arabic*\alph*)}]% 1a), 2b), 3c) ...
	\item Text
	\item Text
\end{enumerate}

\begin{enumerate}[label=\bfseries Punkt \Roman*]% Punkt I, Punkt II, Punkt III ...
	\item Text
	\item Text
\end{enumerate}

\section{Mathe - Beispiele}\label{mathe-bsp}

% \begin{align}	\quad \text{;} \quad \end{align}

\num{12345,678999}


\subsection{Potenzen }\label{sec:potenzen }\index{Potenzen }

allgemein:

\begin{align}
a^n &= a \cdot a \cdot ... \cdot a_n
\end{align}

Multiplikation: (gl.Basis, gl. Exponent)

\begin{align}
a^n \cdot a^m &= a^{n+m} \\
a^n \cdot b^n &= (a \cdot b)^n
\end{align}

Division:

\begin{align}
\frac{a^n}{a^m} &= a^{n-m} \\
a^{-n}          &= \frac{1}{a^n}   \\
a^0 						&= 1 \\
a^1				      &= a
\end{align}

Potenzen potenzieren:

\begin{align}
(a^n)^m         &= a^{n \cdot m} \\
\frac{a^n}{b^n} &= \left(\frac{a}{b}\right)^n
\end{align}

\begin{align}
a^b        &= e^{b \cdot ln \, a} \\
\sqrt[n]{a^m} &= a^{\frac{m}{n}}
\end{align}

(\autoref{code:mathe-out} Mathe).
% Quellcode
\lstset{language=[LaTeX]TeX} % C, [LaTeX]TeX, Bash, Python
\begin{lstlisting}[columns=fullflexible,%numbers=left, frame=l, framerule=0.1pt,%
% ======================
	caption={Quellcode in LaTeX, Mathe},% Anpassen!
	label={code:mathe-out}%
%============================
]
	allgemein:

	\begin{align}
	a^n &= a \cdot a \cdot ... \cdot a_n
	\end{align}

	Multiplikation: (gl.Basis, gl. Exponent)

	\begin{align}
	a^n \cdot a^m &= a^{n+m} \\
	a^n \cdot b^n &= (a \cdot b)^n
	\end{align}

	Division:

	\begin{align}
	\frac{a^n}{a^m} &= a^{n-m} \\
	a^{-n}          &= \frac{1}{a^n}   \\
	a^0 						&= 1 \\
	a^1				      &= a
	\end{align}

	Potenzen potenzieren:

	\begin{align}
	(a^n)^m         &= a^{n \cdot m} \\
	\frac{a^n}{b^n} &= \left(\frac{a}{b}\right)^n
	\end{align}

	\begin{align}
	a^b        &= e^{b \cdot ln \, a} \\
	\sqrt[n]{a^m} &= a^{\frac{m}{n}}
	\end{align}
\end{lstlisting}

\newpage

\section{LaTeX - Befehle}

Textauszeichnung

(\autoref{code:textauszeichnung-out} Textauszeichnung).
% Quellcode
\lstset{language=[LaTeX]TeX} % C, [LaTeX]TeX, Bash, Python
\begin{lstlisting}[columns=fullflexible,%numbers=left, frame=l, framerule=0.1pt,%
% ======================
	caption={Quellcode in \LaTeX: Textauszeichnung},label={code:textauszeichnung-out}]
	\emph{kursiv}
	\textrm{Antiqua}, \textsf{Grotesk}, \texttt{Maschinenschrift},
	\textmd{normal}, \textbf{breiter}, \textup{aufrecht}, \textsl{geneigt},
	\textit{kursiv}, \textsc{Kapitaelchen}
\end{lstlisting}

Schriftgroesse

(\autoref{code:schriftgroesse-out} Schriftgroesse).
% Quellcode
\lstset{language=[LaTeX]TeX} % C, [LaTeX]TeX, Bash, Python
\begin{lstlisting}[columns=fullflexible,%numbers=left, frame=l, framerule=0.1pt,%
% ======================
	caption={Quellcode in \LaTeX: Schriftgroesse},label={code:schriftgroesse-out}]
  \tiny{winzig}, \scriptsize{sehr klein}, \footnotesize{klein},
	\small{klein}, \normalsize{normal}, \large{gross}, \Large{groesser},
	\LARGE{ganz gross}, \huge{riesig}, \Huge{gigantisch}
\end{lstlisting}

 eigene Befehle definieren

(\autoref{code:eigene-Befehle-definieren-out} eigene Befehle definieren).
% Quellcode
\lstset{language=[LaTeX]TeX} % C, [LaTeX]TeX, Bash, Python
\begin{lstlisting}[columns=fullflexible,%numbers=left, frame=l, framerule=0.1pt,%
% ======================
	caption={Quellcode in \LaTeX: eigene Befehle definieren},label={code:eigene-Befehle-definieren-out}]
	\wort{Beispiel}
	\fremdwort{prezioes}
	\abstand{}

	\newcommand{\wort}[1]{\emph{#1}}
	\newcommand{\fremdwort}[1]{\textsf{#1}}
	\newcommand{\abstand}[1]{\vspace{5mm}{#1}}
	\newcommand{\wichtig}[2][red]{\textcolor{#1}{\emph{#2}}}

	quad, qquad, hspace{20mm}, vspace{20mm}
	Wichtig (Optionale Parameter)
	Wort Kursiv u. in Farbe
\end{lstlisting}

Eigene Umgebung

(\autoref{code:Eigene-Umgebung-out} Eigene Umgebung).
% Quellcode
\lstset{language=[LaTeX]TeX} % C, [LaTeX]TeX, Bash, Python
\begin{lstlisting}[columns=fullflexible,%numbers=left, frame=l, framerule=0.1pt,%
% ======================
	caption={Quellcode in \LaTeX: TEigene Umgebung},label={code:Eigene-Umgebung-out}]
	Verwendung: \begin{hinweis}Ein Text.\end{hinweis}

	\newenvironment{hinweis}[1][Hinweis]{%
	  \begin{quote}
	  \color{meinblue}\rule{0.87\textwidth}{1pt}\\%
		\color{black}
	  \textbf{#1:}\\ %
	}{%
		\vspace{1mm}
	  \\\color{meinblue}\rule[5ex]{0.87\textwidth}{1pt}%
	  \end{quote}
	}
\end{lstlisting}

farbige Infobox

(\autoref{code:farbigeInfobox-out} farbige Infobox).
% Quellcode
\lstset{language=[LaTeX]TeX} % C, [LaTeX]TeX, Bash, Python
\begin{lstlisting}[columns=fullflexible,%numbers=left, frame=l, framerule=0.1pt,%
% ======================
	caption={Quellcode in \LaTeX: farbige Infobox},label={code:farbigeInfobox-out}]
	Anwendung: \myInfoBox{Text}

	\newcommand\myInfoBox[1]{%
	  \begin{quote}
		\fcolorbox{meinblue}{meingrey}{%
			\parbox{0.85\textwidth}{%
				\textbf{Hinweis:}\\%
				#1
			}
		}
	\end{quote}
	}
\end{lstlisting}

farbige Listenbox

(\autoref{code:farbigeListenbox-out} farbige Listenbox).
% Quellcode
\lstset{language=[LaTeX]TeX} % C, [LaTeX]TeX, Bash, Python
\begin{lstlisting}[columns=fullflexible,%numbers=left, frame=l, framerule=0.1pt,%
% ======================
	caption={Quellcode in \LaTeX: farbige Listenbox},label={code:farbigeListenbox-out}]
	Anwendung:
	\myListenBox {
		\item Listenpunkt
		\item Listenpunkt
		\item Listenpunkt
	}

	\newcommand\myListenBox[1]{%
		\begin{quote}
			\fcolorbox{meinblue}{white}{%
				\parbox{0.85\textwidth}{%
					% Inhalt
					\textbf{Liste: }
					\begin{itemize}[label=$\square$]%checkbox
						#1
					\end{itemize}
				}
			}
		\end{quote}
	}
\end{lstlisting}

(\autoref{code:begriffe-out} Begriffe).
% Quellcode
\lstset{language=[LaTeX]TeX} % C, [LaTeX]TeX, Bash, Python
\begin{lstlisting}[columns=fullflexible,%numbers=left, frame=l, framerule=0.1pt,%
% ======================
	caption={Quellcode in \LaTeX: Begriffe},label={code:begriffe-out}]
	Referenz: 				siehe~\ref{sec:abschnitt}.
	Zitat: 						siehe~\cite{Bos15}.
	Textauszeichnung: \wort{Beispiel}, \fremdwort{Fremdwort}
	Textabstand: 			\abstand{}
	Zahl Einheit: 		1\,l   z.\,B.
	nicht trennpaares Leerzeichen: ~
	Sonderzeichen: 		\& \% \$ \%
	Webadresse:				\url{http://www.LaTeXbuch.de}

	twoside=true		mschaltung Zweiseitig/Einseitiges Layout
	fontsize=12pt		Schriftgröße
	BCOR=10mm			  Bindekorrektur
	parskip					Gibt an wie neue Absätze gekennzeichnet werden.
								Hier empfohlene Beispielwerte:
									false:	Einzug der ersten Zeile
									half:		vertikaler Abstand von einer halben Zeile
									full:		vertikaler Abstand von einer Zeile
	paper=a4				DIN A4 Papier
	toc=listof			Im Inhaltsverzeichnis werden verzeichnisse wie Abbildungsverz.
								aufgenommen, wenn nicht gewünscht toc=nolistof
	toc=bib				  Literaturverzeichnis ohne Nummer im Inhaltsverzeichnis, oder nöchste zeile
	bibliography=totocnumbered	Literaturverzeichnis mit Nummer im Inhaltsverzeichnis, totoc ohne nummer
	open=right			Ein neues Kapitel fängt immer auf einer rechten Seite an, sonst open=any
	numbers=noenddot Nach DUDEN Werden Gliederungsnummern ohne Punkt am Ende gesetzt.
	headinclude			Kopfzeile zählt mit zum Graubereich der Seite
	headlines=2			zweizeilige Kopfzeile
	footexclude			Fußzeile enthält z.B. nur die Seitenzahl zählt deshalb nicht zum Graubereich der Seite
	pagesize=auto		Sorgt dafür, dass das PDF auch die richtige Größe hat
	version=last		welche Version des KOMA-Scripts verwendet werden soll
\end{lstlisting}

\newpage

\section{Quellenangaben}

Quelle \textcite{schlosser_latex:2016}

Quelle Text\footfullcite{schlosser_latex:2016}

Quelle \cite{schlosser_latex:2016}
%============================
%============================
%============================